% TEMPLATE for Usenix papers, specifically to meet requirements of
%  USENIX '05
% originally a template for producing IEEE-format articles using LaTeX.
%   written by Matthew Ward, CS Department, Worcester Polytechnic Institute.
% adapted by David Beazley for his excellent SWIG paper in Proceedings,
%   Tcl 96
% turned into a smartass generic template by De Clarke, with thanks to
%   both the above pioneers
% use at your own risk.  Complaints to /dev/null.
% make it two column with no page numbering, default is 10 point

% Munged by Fred Douglis <douglis@research.att.com> 10/97 to separate
% the .sty file from the LaTeX source template, so that people can
% more easily include the .sty file into an existing document.  Also
% changed to more closely follow the style guidelines as represented
% by the Word sample file. 

% Note that since 2010, USENIX does not require endnotes. If you want
% foot of page notes, don't include the endnotes package in the 
% usepackage command, below.

% This version uses the latex2e styles, not the very ancient 2.09 stuff.
\documentclass[letterpaper,twocolumn,10pt]{article}

\usepackage{usenix,epsfig,endnotes}
\usepackage{amsmath}
\usepackage{color}
%\usepackage{hyperref}
\usepackage{algorithm}
\usepackage{algpseudocode}
\usepackage{grumble}
\usepackage{pseudocode}
\usepackage{macro}
\usepackage{graphicx}
\usepackage{subfigure}
\usepackage{url}
\usepackage{multirow}
%try to reduce the space between items
\usepackage{enumitem}
\setlist{nolistsep}

%\renewcommand{\baselinestretch}{.97}
%\addtolength{\textheight}{0.0cm}
%\addtolength{\textwidth}{0.0cm}
\addtolength{\intextsep}{-0.2cm}% : space left on top and bottom of an in-text float.
\addtolength{\textfloatsep}{-0.3cm}% is \textfloatsep for 2 column output.
\addtolength{\abovecaptionskip}{-0.2cm}%: space above caption
\addtolength{\belowcaptionskip}{-0.2cm}%: space below caption

\let\oldsubsection\subsection
\renewcommand{\subsection}[1]{%
  \vspace{-0.5cm}\oldsubsection{#1}\vspace{-0.1cm}%
}

\let\oldsection\section
\renewcommand{\section}[1]{
  \vspace{-0.5cm}\oldsection{#1}\vspace{-0.1cm}%
}


\begin{document}

%don't want date printed
\date{}

%make title bold and 14 pt font (Latex default is non-bold, 16 pt)
%\title{\Large \bf Wonderful : A Terrific Application and Fascinating Paper}
\title{TBD}

%for single author (just remove % characters)
\author{%\fontsize{10}{12} \selectfont
TBA
%%%{\rm Cheng Li$^\ddag$, Jo\~{a}o Leit\~{a}o$^\dag$,  Allen Clement$^\ddag$, Nuno Pregui\c{c}a$^\dag$, Rodrigo Rodrigues$^\dag$, Viktor Vafeiadis$^\ddag$}\\
%%%$\dag$ CITI-FCT, $\ddag$ MPI-SWS
%% {\rm Cheng Li, Allen Clement, Viktor Vafeiadis}\\
%% MPI-SWS
%% \and
%% {\rm Jo\~{a}o Leit\~{a}o, Nuno Pregui\c{c}a, Rodrigo Rodrigues}\\
%% CITI-FCT
%% % copy the following lines to add more authors
%% \and
%% {\rm Allen Clement}\\
%% MPI-SWS
%% \and
%% {\rm Nuno Pregui\c{c}a}\\
%% CITI-FCT  
%% \and
%% {\rm Rodrigo Rodrigues}\\
%% CITI-FCT 
%% \and
%% {\rm Viktor Vafeiadis}\\
%% MPI-SWS
} % end author

\maketitle

% Use the following at camera-ready time to suppress page numbers.
% Comment it out when you first submit the paper for review.
\thispagestyle{empty}

\input{abstract}

\section{Introduction}
\label{sect:introduction}
TBA

\section{Related work}
\label{sect:related}

TBA

%\section{Implementation details and challenges}
\label{sect:implement}

TBA

\section{Evaluation}
\label{sect:evaluation}

TBA

\section{Conclusion}
\label{ch:por:sect:conclude}
In this paper, we proposed a research direction for building fast and consistent
geo-replicated system that employs a minimal amount of coordination in
order to achieve both invariant preservation and state convergence. 
To this end, we first defined a new generic consistency model called \PRCN, which maps
consistency requirements to fine-grained restrictions over pairs of operations. Second,
we developed a static analysis to infer for a given application
a minimal set of restrictions for ensuring the two previously mentioned properties, in which
no restrictions can be removed or no new restrictions need to be added. Third, we built an efficient
coordination service called \coordtool, which offers two different coordination protocols
suiting for different workloads. Our evaluation of running RUBiS with different
setups shows that the joint work of \PRCN\ and \coordtool\ significantly
improves system performance of geo-replicated systems.



\let\section\oldsection

\let\oldbibliography\thebibliography
\renewcommand{\thebibliography}[1]{%
  \oldbibliography{#1}%
  \setlength{\itemsep}{-5pt}%
}

{
\renewcommand{\baselinestretch}{.9}
 \vspace{-0.5cm}
\bibliographystyle{acm}
\bibliography{main}}

\renewcommand{\baselinestretch}{1.0}
\vspace{-0.5cm}
\theendnotes

\end{document}

