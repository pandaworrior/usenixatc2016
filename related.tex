\section{Related work}
\label{ch:por:sect:related}
We summarize most relevant related work and compare it against our \PRCN\ framework in the following
categories.

\paragraph{Consistency models.} In the past decades, 
many consistency proposals have been focusing on the reduction in coordination
among concurrent operations to improve scalability 
in replicated systems~\cite{Ladin1992LazyReplication,
Sovran2011PSI, Li2012RedBlue, Li2014SIEVE, Alvaro2014Blazes, Alvaro2011Bloom, Zhang2015InconsRep}. 
However, they only allow the programmer to choose from a limited number of
consistency levels that they support, such as strong, causal or eventual consistency.
Unlike these approaches, \PRCN\ offers a fine-grained tunable tradeoff
between performance and consistency using the 
visibility restrictions between pairs of operations to express consistency semantics.
In addition, most previous proposals~\cite{Alvaro2014Blazes, Ladin1992LazyReplication,
Sovran2011PSI, Alvaro2011Bloom} only take into account operation commutativity
to determine the need for coordination, instead of invariant preservation, which
is analyzed in our solution.

In the family of consistency proposals concerning application-specific invariants,
Bailis et al.~\cite{Bailis2014Avoid} proposed I-confluence
to avoid coordination by determining if a set of transactions are I-confluent
w.r.t database integrities, i.e., 
integrity constraints might be violated if they were executing without coordination.
Indigo~\cite{Balegas2015Eventual} defines consistency as 
a set of invariants that must hold at any time, and presents a set of mechanisms to enforce these invariants 
efficiently on the top of eventual consistency. Similar to Indigo, warranties~\cite{Liu2014Warranties} 
map consistency requirements to a set of assertions that must hold in a given period of time,
but it needs to periodically invalidate assertions when updates arrive. In contrast to
these approaches, PoR consistency takes an alternative approach 
by modeling consistency as restrictions over operations.

There also exist a few proposals which map consistency semantics to 
the ordering constraints defined over pairs of operations. For example, Generic Broadcasting 
defines conflict relations between messages for fast message delievery,
which are analogous to visibility restrictions used in our solution~\cite{Pedone99genericbroadcast}. 
Most recently, a concurrent work proposed by Gotsman et al. 
encoded the {\tt conflict relation} concept into 
a proof system, which enables to analyze if consistency 
choices expressed into conflict relations
meet the target properties~\cite{GotsmanConsistencyReason}. Our approach differs from 
all these consistency proposals in the following aspects. First, we have a different formalism that 
captures new situations that lead to invariant violations, such as the 
possibility of three different types of requests being necessary to trigger such cases. Second, 
we provide programmers with the ability to infer a
minimal set of (fine-grained) restrictions to achieve state convergence
and invariant preservation. Third, we explore the possiblity of using different 
coordination protocols to enforcing restrictions efficiently.

\paragraph{Paxos and its variants.} State machine replication~\cite{Schneider1990RSM} is
a standard technique to make a set of servers behave like a single machine.
Paxos~\cite{Lamport1998Paxos}, one of the classic algorithms that implement state machines,
forces every replica to process a set of requests in the same
sequential order. In order to reduce the number of message exchanges for achieving distributed
consensus, several variants of Paxos have been proposed. Fast Paxos~\cite{Lamport2006FastPaxos} aims
at improving latency by allowing every replica to propose values but suffers from high latency when concurrent
proposals occur. To avoid the penalty introduced by collisions, some other variants of Paxos explore 
operation semantics to take into account a weaker guarantee that
not all operations are needed to be totally ordered~\cite{Lamport2005GeneralizedPaxos, Moraru2013EPaxos}. 
Generalized Paxos (GPaxos) allows replicas to execute a set of operations in
different orders as long as operations commute w.r.t each other; however, it still has to resort to 
the classic Paxos algorithm~\cite{Lamport1998Paxos} when the leader notices two concurrent non-commuting requests~\cite{Lamport2005GeneralizedPaxos}.
Egalitarian Paxos (EPaxos) takes as input a set of pre-defined constraints, each of which defines a dependency 
between a pair of operations, and enables each replica to order two concurrent conflicting requests according
to their apriori dependency relation~\cite{Moraru2013EPaxos}. 
%EPaxos commits requests along either a fast or slow path. The fast path corresponds to one
%round of communication, if a request is not con
%icting with any other concurrent requests. If two requests are con
%icting, due to the fixed ordering constraint, one request can
%still be committed along the fast path, while the other request
%must follow a slow path, where a second round of
%communication between leader and participants is triggered
%to enforce that all replicas to eventually learn that the slow
%path request depends on the fast path one~\cite{Moraru2013EPaxos}.
A major difference between our work and these Paxos variants is that we
develop an analysis to extract pairs of conflicting operations by considering the impact of
concurrent executions on achieving state convergence and invariant preservation. 
Furthermore, all these protocols only reduce the number of communication steps, but
still require to talk to a large quorum of replicas. In contrast, in our work,
operations that are not confined by conflicting relations can be first accepted
in a single replica and later asychronously replicated to other replicas.


\paragraph{Efficient transaction processing.} Some other work focuses on how to reduce
coordination in transaction processing. For example, transaction chopping~\cite{Shasha1995Chopping} and
Lynx~\cite{Zhang2013TransactionChain} suggest that breaking large transactions into smaller pieces
can improve performance, and they design analysis algorithms for chopping transactions
without sacrificing serializibility. While this work has been done merely by checking
conflicts in read/write sets between pairs of transaction pieces, we design a comprehensive
and fine-grained analysis concerning commutativity
and invariant preservation for avoiding coordination when possible. These techniques are also
orthogonal to our proposal so that we can apply them to prune out the non-critical code sections prior
to running our analysis. 
%In addition, Roy et al. proposed to reduce inter-node communication
%for executing distributed transactions~\cite{Roy2015Homeostasis}. In contrast, we explore an alternative solution to improve
%system scalability by fully replicating state.
%these solutions so that their spurious conflicts can be further reduced. Some other work takes an approach of
%weakening isolation for transactions~\cite{Xie2014Salt}.

%\paragraph{Efficient coordination.} Zookeeper?

%\cheng{Missing points:  ~\cite{Roy2015Homeostasis}, ~\cite{Alvaro2011Bloom}, put into the main intro as well, ~\cite{Conway2012Logic}, ~\cite{Garcia-Molina1983Semantic} put into intro, ~\cite{Balegas2015FastInvariant}}


